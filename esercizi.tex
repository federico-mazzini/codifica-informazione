\documentclass[11pt]{exam}

\usepackage[utf8]{inputenc}
\usepackage{lmodern}
\usepackage{amsmath}
\usepackage{amssymb}
\usepackage{array}
\usepackage{graphicx}

\begin{document}


\section{Binario}
\subsection{Decifratore di Codici}

In questo esercizio ogni lettera dell'alfabeto viene rappresentata dal suo numero corrispondente (A=1, B=2, ..., Z=26, spazio vuoto=0).  
Il numero viene poi convertito in \textbf{binario} utilizzando il minor numero di bit necessari. Ricostruire la frase segreta a partire dai codici binari forniti.

\emph{Messaggio da decifrare}

\begin{center}
\begin{tabular}{c}
01001101 \\
01101111 \\
01101111 \\
01101110 \\
00000000 \\
01010010 \\
01101001 \\
01110110 \\
01100101 \\
01110010 \\
\end{tabular}
\end{center}

\subsection{Domande}
\begin{questions}

\question In un sistema di numerazione posizionale in base 7, indicare il massimo valore rappresentabile con 3 cifre (sempre in base 7). Quanto vale lo stesso numero in base 10?

\question Che numero decimale rappresenta il numero binario 11111111? N.B. non serve utilizzare il metodo di conversione visto in classe.

\question Quanti valori distinti è possibile rappresentare con 5 bit?

\question Nel sistema binario, qual è il valore del bit più significativo in un numero di 6 bit?

\question Rappresentare il numero $25_{10}$ in binario utilizzando esattamente 8 bit.

\question Rappresentare il numero $72_{10}$ in binario utilizzando esattamente 8 bit.

\question Qual è il minor numero di bit necessari per rappresentare il numero 4356?

\question Un numero scritto in base 4 è $3210_4$. Converti il numero in base 10 mostrando i pesi delle cifre.

\end{questions}

\subsection{Messaggio segreto in bianco e nero}

Dati i seguenti numeri, convertirli in binario e riportarli su una griglia 6 x 9, colorando quando le caselle con bit = 1. 
\begin{center}
\begin{tabular}{c}
198 \\
297 \\
260 \\
258 \\
297 \\
198 \\
\end{tabular}
\end{center}

Che cosa appare sul foglio?


\begin{center}
\includegraphics[width=0.1\textwidth]{images/imm_bianconero_con_bit.png}

\emph{Esempio di colorazione di una griglia 6 x 6}
\end{center}



\newpage
\section{Esadecimale}

\subsubsection*{Domande}

\begin{questions}
\question In codifica esadecimale, scrivere i numeri successivi ai seguenti:
\begin{parts}
    \part $F$
    \part $1A$
    \part $2F$
    \part $3C$
    \part $AF$ 
\end{parts}

\question Quanti bit sono necessari per rappresentare i valori possibili di una singola cifra esadecimale?

\question Quante cifre esadecimali servono per rappresentare tutti i valori possibili di un byte (8 bit)?

\question Quanti valori distinti si possono rappresentare con 1 cifra esadecimale? E con 2 cifre? E con 3 cifre?

\question Qual è il minor numero di cifre esadecimali necessarie per rappresentare il numero decimale $10000_{10}$? 
\end{questions}

\subsubsection*{Conversioni}

\begin{questions}

\question Dato il numero esadecimale $1A3$, convertirlo in decimale mostrando i calcoli con i pesi delle cifre.

\question Dato il numero esadecimale $CD3$, convertirlo in decimale passando per il binario.\\Quindi $esadecimale \rightarrow binario \rightarrow decimale$.

\question Dato il numero decimale $255_{10}$, convertirlo in esadecimale passando per il binario.\\Quindi $decimale \rightarrow binario \rightarrow esadecimale$.

 
\question Dato il numero esadecimale $CF79E_{16}$, convertirlo in binario utilizzando la tabella di corrispondenza (pagina 10 dispense). Quanti byte servono per rappresentarlo?.

\question Dato il numero binario 11111100010101000, convertirlo in esadecimale utilizzando la tabella di corrispondenza (pagina 10 dispense).

\end{questions}



\newpage
\section{Codifica delle immagini}


Gli esercizi seguenti permettono di applicare in modo concreto i concetti di risoluzione, pixel e profondità di colore.

\begin{questions}

    \question Una piccola icona ha dimensione {100 $\times$ 100} pixel.  
    \begin{parts}
        \part Quanti pixel totali contiene?
        \part Se è in bianco e nero (1 bit per pixel), quanti byte servono per memorizzarla?
        \part Se è a 24 bit per pixel, quanti byte servono?
    \end{parts}

    \question Considerando un'immagine Full HD, con risoluzione {1920 $\times$ 1080} pixel
    \begin{parts}
        \part Quanti pixel totali contiene l'immagine?
        \part Se l'immagine è in bianco e nero (1 bit per pixel), quanti byte sono necessari per memorizzarla?
        \part Se l'immagine è a colori con profondità di colore di 24 bit per pixel (modello RGB, 8 bit per ogni canale di colore), quanti byte sono necessari per rappresentare l'immagine? A quanti megabyte circa corrisponde questa dimensione?
    \end{parts}
\end{questions}

\end{document}